% This file contains the abstract part of your thesis - in English and
% in Hebrew (within \abstractEnglish and \abstractHebrew respectively).
%
% Notes:
% - This file uses the UTF-8 character set encoding for the Hebrew
%   text not to get garbled. Keep it that way.
% - Assuming your thesis is mainly in English, Graduate School 
%   regulations mandate the following lengths for the abstracts:
%
%      Language    Min. Length   Max. Length
%     ---------------------------------------
%      English       200 words     500 words
%      Hebrew        500 words   2,000 words
%
%   so that the Hebrew abstract typically has some content from
%   the English introduction and an overview of the results, not
%   present in the English; it is not just a translation.

\abstractEnglish{

In natural microbial communities, antagonistic interactions among members of these are important drivers of community evolution and ecological dynamics. The natural environments in which those communities exist often involve spatial structure, which could be a key factor in evolutionary processes as well as in expansion dynamics when growing into new habitats, yet it is often neglected in microbial community experimentation and simulations. In this thesis, we will present two projects, one experimental and one modeling a biological system \textit{in silico}. In the first project, we investigated selection and evolution of antibiotic producers in microdroplets, a microfluidic environment that effectively forms small, distinct environments to limit interference and allow for antibiotic producers to gain benefits. We were able to modify an existing microfluidic system to allow stable production and long-term incubation of microdroplets, containing interacting bacterial communities. We aimed to create a selection mechanism for antibiotic-producing bacteria as a result of the artificial spatial structure of the microdroplets. In contrast to a well-mixed environment, the benefit of producing an antibiotic is not shared in this case, and therefore, a selective advantage for the producing bacteria exists. Unfortunately, after extensive screening of target strains and choosing a producer-target pair, we were unable to grow the antibiotic producer properly in droplets and showed that it only produces a toxin on solid growth media but not in the liquid environment required for droplets. In the second project, we used a simple reaction-diffusion model to simulate bacteria-phage interactions in a spatially structured environment where phage-sensitive and phage-resistant bacteria are competing for resources. We focused on the impact of resistance and the different properties of resistant bacteria on the system. The model exhibits traveling waves, resulting in a quasi-steady state traveling through space. We show how in such a traveling wave scenario, phage-resistant bacteria can protect sensitive bacteria through an indirect slowdown of phage migration. This is in stark contrast to a simulated well-mixed chemostat environment where competition with phage-resistant bacteria can lead to extinction of phage-sensitive bacteria. We show that the effect depends on the speed of resistant bacteria and, exploring a possible parameter space, we were able to suggest which parameters are critical for the existence of the described protective effect. Previous models report a similar, much weaker effect in the absence of resistant bacteria. Using nutrient limitation, we observed a weak effect without resistant bacteria, but this effect is dramatically enhanced when phage-sensitive and phage-resistant bacteria are in competition, confirming the need for phage-resistant bacteria to observe a strong protective effect. Together, these studies help shed light on the complex spatio-temporal dynamics of antagonistically interacting microbial species. 

% At this point you write the abstract of your work, in the main language in which it is written (in this template - English). Graduate school regulations require the abstract to constitute an independent whole and be understood to a reader with general knowledge of the field. Use complete sentences and make few or no citations. Do not refer to the main body of the work and do not use uncommon shorthand, symbols and terms unless you have room for explaining them. The English abstract should be between 200 and 500 words long.

% So this should contain a few more paragraphs... we'll fill them using some placeholder text (in Latin):

% \lipsum[10-12]

} % end of English abstract


\abstractHebrew{

% Note that certain commands don't work that well in Hebrew "mode".
% If this happens to you, try wrapping the command within a
% \textenglish{ } - that may (or may not) help.

Add abstract in Hebrew

% כאן יבוא תקציר מורחב בעברית (כאשר שפת החיבור העיקרית היא אנגלית). היקף התקציר יהיה \textenglish{1000-2000} מילים. התקציר יהווה שלמות בפני עצמו ויהיה מובן לקורא בעל ידיעות כלליות בנושא.

% בית הספר ללימודי מוסמכים מנחה מספר הנחיות לגבי התקציר בעברית:
% \begin{itemize}
% \item על התקציר להיכתב במשפטים מקושרים שלמים.
% \item בדרך-כלל אין לציין בתקציר מקורות ספרותיים וציטוטים.
% \item אין להתייחס למספר של פרק, סעיף, נוסחה, ציור או טבלה שבגוף החיבור, ואין להשתמש בקיצורים, סמלים ומונחים לא מקובלים, אלא אם יש בתקציר די מקום לזיהויים.
% \end{itemize}

% לעתים יש בכל-זאת יש צורך לכלול פקודה הכוללת קישור פנימי או חיצוני בתוך התקציר העברי; במצבים כאלו כדאי דרך-כלל לעטוף את הפקודה היוצרת את הקישור בתוך פקודת \textenglish{\texttt{\textbackslash{}textenglish\{\}}} כדי למנוע כל מיני פורענויות בלתי-רצויות, כגון כישלון בהידור קובץ ה-\textenglish{PDF} או שימוש בגופן העברי באופן אשר עלול שלא להנעים לעין. לדוגמה: נניח שיש לנו צורך לצטט מקור ביבליוגרפי. אם נעשה זאת סתם-כך: \textenglish{\texttt{\textbackslash{}cite\{Hoeffding\}}}, נקבל: \cite{Hoeffding}; אם נעטוף את פקודת הציטוט, כך: \textenglish{\texttt{\textbackslash{}textenglish\{\textbackslash{}cite\{Hoeffding\}\}}}, נקבל \textenglish{\cite{Hoeffding}} (כפי שהציטוטים נראים גם בטקסט באנגלית).

% \subsection*{\texthebrew{תת-חלק בתקציר המורחב}}

% תוכן מקוצר לגבי נושא מסוים. התייחסות ל\emph{מושג} מסוים שהחיבור בוחן. וכולי וכולי.


% \subsection*{\texthebrew{נקודה מעניינת לגבי העמודים בעברית}}

% שימו לב כי העמודים בעברית אמורים להיות מיוצרים בסדר ה''הפוך'', הווה אומר העמוד האחרון בקובץ ה-\textenglish{PDF} הוא הכריכה העברית, לפניו השער העברי, ודפי התקציר צריכים להופיע בסדר הפוך (וכן במספור רומי, לפי נהלי הטכניון). כך אם נתבונן במספר שבתחתית עמוד זה \textenglish{---} אשר צריך להיות העמוד הראשון בתקציר-המורחב מבחינת רצף התוכן, והינו העמוד האחרון מבין עמודי התקציר-המורחב אחרון בקובץ ה-\textenglish{PDF} \textenglish{---} נמצא את המספר \textenglish{i} ...

% \newpage

% ... ואילו עמוד זה של התקציר-המורחב בעברית \textenglish{---} שהינו העמוד השני בתקציר-המורחב מבחינת רצף התוכן, ונמצא ראשון בקובץ ה-\textenglish{PDF} \textenglish{---} ממוספר ב-\textenglish{ii}. המטרה במספור בסדר ה"הפוך" היא, שבעת ההדפסה לא יהיה צורך להפוך דפים, לשנות את סדרם וכולי \textenglish{---} רק להדפיס ולכרוך.

 } % end of Hebrew abstract
