\chapter{Conclusion and open questions}
\label{chap:conclusion}

In this project, we built a model to describe bacteria-phage interactions in an expanding traveling wave and studied it in a one-dimensional scenario. We modified the model to allow the study in a zero-dimensional chemostat-like scenario. 
Comparing these two scenarios, our model reveals that in a spatial environment, in complete opposite to a well-mixed environment, adding resistant bacteria to the system can increase the amount of sensitive bacteria in a traveling wave.
Studying the underlying causes for this protective effect, we found that the decoupling of phage and bacterial front causes this protective effect. Through studying different parameter regimes we could determine realistic conditions in which this decoupling and therefore the protective effect is observed.
Further studies should study the effect of parameter changes on well-mixed environments and studying different experimentally obtained parameter choices to adjust the model even more to experimental studies.
