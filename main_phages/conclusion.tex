\chapter{Conclusion and open questions}
\label{chap:conclusion}

In this project, we built a model to describe bacteria-phage interactions in an expanding traveling wave and studied it in a one-dimensional scenario. We modified the model to allow the study in a zero-dimensional chemostat-like scenario. 
Comparing these two scenarios, our model reveals that in a spatial environment, in complete opposite to a well-mixed environment, adding resistant bacteria to the system can increase the amount of sensitive bacteria in a traveling wave.
Studying the underlying causes for this protective effect, we found that the decoupling of phage and bacterial front causes this protective effect. Through studying different parameter regimes we could determine realistic conditions in which this decoupling and therefore the protective effect is observed.


% This kind of chapter can include may different things (or only some of them):
% \begin{itemize}
% \item Discussion of results
% \item Conclusions from the results or from the process in general
% \item Open questions for future research, resulting from the research performed or from the results obtained
% \end{itemize}

% But not things like the bibliography or other back matter which is generated outside of this chapter.


% \section{Some conclusion}

% Here is what I conclude.

% \section{Some open questions}

% \paragraph{A question in brief.} In \autoref{chap:firstchap} we explored a certain subject, but what about this-or-that idea? Perhaps it is worth exploring. Can one produce interesting results?

% \paragraph{A second question in brief.} A broader exposition of the question and indications of directions or ideas regarding its resolution.

