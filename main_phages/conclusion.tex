\chapter{Conclusion and open questions}
\label{chap:conclusion}

In this project, we have built a model to describe bacteria-phage interactions in an expanding one-dimensional traveling wave scenario as well as in a well-mixed, liquid, chemostat scenario. Comparing these two scenarios, our model reveals that in a spatially structured, one-dimensional environment, in complete opposite to a well-mixed environment, adding resistant bacteria to the system can maintain and even increase the amount of sensitive bacteria in the system over time.
Studying the underlying reason for this protective effect, we found that the decoupling of phage and bacterial front results in a widening of the sensitive traveling wave over time and in an increase of the amount of sensitive bacteria over time. By comparing with a scenario without resistant bacteria and direct nutrient limitation, we could show that resistant bacteria are indeed necessary for the effect to occur as the effect is not observed in a nutrient-poor environment. Through studying different parameter regimes of crucial parameters, we could determine realistic conditions in which this decoupling and the protective effect occurs.
Further studies should study the effect of parameter changes on well-mixed environments and make the model reflecting more closely real world conditions. It remains to be seen if this protective effect occurs also in more complex competitions involving more bacterial species and phages.
