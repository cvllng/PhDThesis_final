\part{Phage-resistant bacteria protect phage-sensitive bacteria in a spatial environment}
\chapter{Introduction}
\label{chap:intro}

This chapter introduces the key biological concepts for this project, namely bacteriophages and their interaction with bacteria. Furthermore, this chapter introduces the importance of spatial structure in natural environments and its impact on ecological dynamics. Lastly, we introduce reaction-diffusion equations, the class of computational models used in this project to simulate dynamics in a spatially structured environment and we highlight some applications of such models.

\section{Bacteriophages}
Bacteriophages (phages) are viruses infecting bacteria. They are binding a receptor on the bacterium, then the phage-DNA is injected in the bacterium. The replication machinery in the bacterium then replicates the phage DNA and after translation and transcription, new phages are assembled inside the bacterium. After some time, the host cell is lysed and new phages released~\cite{Clokie2008-mc}. This replication cycle of phage can vary slightly and two types of phages are defined. Lytic phages which lyse the cell immediately and lysogenic phages where the genomic material is first incorporated into the bacterial genome and only after some time or upon some trigger, phage replication and lysis is activated~\cite{Young1992-nu, Howard-Varona2017-gi}. In this project we will focus on lytic phages. Bacteria can develop resistance against phage through many mechanisms such as blocking the receptor, cleaving phage DNA upon entry or modifying the entry channels through which genomic material enters~\cite{Labrie2010-rl}. For this project we assume that phages cannot permanently bind to resistant bacteria while we do not specify the exact resistance mechanism.

\section{Experimental spatial expansion}
Work on ecological interactions of bacteria and bacteriophages is mostly performed in well-mixed, liquid environments due to the simpler handling and homogeneity~\cite{Bull2018-ha}. One of the most prominent results in well-mixed environments is the rapid overtake of the dynamics by a phage-resistant bacterial genotype out-competing phage-sensitive ancestors without the phages being able to overcome this resistance~\cite{Lenski1985-wb}.
Assuming, this would also happen in natural communities, leads to a paradox: Why do phages still exist when phage-resistant bacteria can easily overtake the population? Looking at spatial structure, as one stark difference between these results and most natural communities, recent work in our lab and from others has shown that sensitive bacteria and phage exhibit more complex dynamics during expansion on a plate with survival of sensitive bacteria at the front and further continuous evolution of bacteria and phages resulting in a highly diverse community~\cite{Shaer-Tamar2022-cq, Marchi2025-yu, Ping2020-vd}. So, spatial structure seems to be one of the key factors to ensure diversity of populations.

\section{Modeling of spatial expansion}
To understand the underlying processes in such spatial expansion experiments, mathematical models and simulations are used. Commonly, models are separated into two major classes, single agent models where each bacterium and phage represent single agents and through this explicit modeling, single details such as exact position in the system can be measured. Examples using such models include explicit modeling of phage plaque formation~\cite{Valdez2025-io}. These models, as they are resolving single cell dynamics, are computationally expensive and, therefore, limited to small population sizes~\cite{Nagarajan2022-rv}. Consequently, they are less commonly used when studying larger population dynamics. The second class are continuous models, often using differential equations, where densities are modeled rather than single bacteria or phages. These are more coarse but can capture dynamics and are computationally more feasible for larger systems~\cite{Succurro2018-if}. In this project, we are using a system of coupled~\gls{pde} to model the system. As we are describing spatial expansion and the interaction between components, our model falls in the class of reaction-diffusion models. 
Reaction-diffusion models are a class of models used to model spatial expansion processes. They are characterized by a diffusion part, describing expansion in space and a localized reaction part, describing the coupling to other components or to the environment. A well known reaction-diffusion equation in biological contexts is the Fisher-KPP-equation~\cite{Fisher1937-rd} describing e.g. the expansion of a growing population $A$ in an empty space. It is given by:
\begin{equation}
    \frac{\text{d}A}{\text{d}t} = D_A \frac{\partial^2A}{\partial x^2} + \lambda_A A (1-A)
\end{equation}
where $D_A$ is the diffusion coefficient of $A$ and $\lambda_A$ is the replication rate of $A$.

\section{Modeling bacteria-phage interactions in a spatially structured environment}

Several models of bacteria-phage interactions take spatial structure into account with various spatial structures. Biofilms were simulated and it was shown that phage-resistant bacteria in a biofilm can create niches in which phage-sensitive bacteria are protected from phage infection of the biofilm~\cite{Simmons2020-cc}. Phage plaque formation on an existing bacterial population is well studied and models explain the mechanistic creation of plaques~\cite{Valdez2025-io, Smith2011-ae, Yin1992-th}. Furthermore, the impact of different growth media with different degrees of spatial structure was studied using models and it was shown that phages and bacteria are co-transported in more structured environments~\cite{Wang2024-ye}. Most similar to the experiments conducted in our lab~\cite{Shaer-Tamar2022-cq}, the effect of hitchhiking enabling phages to stay at a bacterial front despite their lower expansion speed was studied~\cite{Ping2020-vd} and most recently, the invasion of an existing sensitive bacterial population by a wave of phages was mathematically described and it was shown that a novel selection mechanism for optimal speed exists if multiple traveling waves are coupled~\cite{Claydon2021-cu}. However, the impact of resistant bacteria on an expanding population in spatially structured environments, relevant in experimental settings such as the experiments carried out in our group~\cite{Shaer-Tamar2022-cq}, was not systematically studied, to the best of our knowledge.