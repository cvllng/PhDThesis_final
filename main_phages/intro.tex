\part{Simulating bacteria-phage interactions}
\chapter{Introduction}
\label{chap:intro}
% do we need to add TOC lines?

%\begin{figure}
%  \centering
%  \includegraphics[width=0.75\textwidth]{main/graphics/a_blowup.pdf}
%  \caption{This is a caption}
%\end{figure}

Here you can introduce the field, survey past results, give context, use citations of course... (e.g. \cite{CLR}). It is probably worthwhile to clarify the goals or targets of the research and describe the process, unless this is done later.

You can also introduce \emph{a key concept} (or rather, several) without formally defining them until later on.

\subsection*{An unnumbered subsection}

You may want to break up the intro into parts with titles. Subsectioning without numbering is an option you might want to consider.

Some people include a specific section overviewing the results ("In Chapter so-and-so, we will see how etc.") which is also a way of describing the structure of the thesis. But this is not necessary.

\subsection*{Thesis options and appearance}

Please note that the \texttt{iitthesis} class has several options when you use it, such as:
\begin{itemize}
\item \texttt{fullpageDraft} to avoid the margins necessary for proper binding when you make the final print
\item \texttt{beforeDefense} makes the personal acknowledgements invisible; use this to print the copies you submit initially to the grad school for sending to the opponent panel, i.e. thesis readers (who shouldn't see those parts). For the final submission, after having successfully defended --- drop this option. 
\item \texttt{noabbrevs} no notation \& abbreviations list will be included in the thesis.
\end{itemize}

\subsection*{Hebrew font}

The \texttt{iitthesis} document class uses the David CLM font family for Hebrew text. CLM is a shorthand for ``Culmus'' (\texthebrew{קולמוס}) --- the name of a freely-available Hebrew font package. It may be bundled with your LaTeX distribution, or otherwise, must be available as system fonts. If you're missing the Culmus fonts, try adding an appropriate package from your LaTeX distribution or system distribution; alternatively, you might want to visit the Culmus project page at \url{http://culmus.sourceforge.net/} and download and install the fonts manually.

\subsection*{Setting thesis meta-data and publication information}

The document class used to generate this document defines several commands you can use to set information  regarding your thesis, which is used in the title pages and elsewhere in the front matter.  Every (or almost every) command has an English and a Hebrew variant, with a \texttt{English} or \texttt{Hebrew} suffix to the command name. Examples:
\begin{itemize}
\item \verb|\titleHebrew|, \verb|\titleEnglish|
\item \verb|\authorHebrew|, \verb|\authorEnglish|
\item \verb|\JewishDateHebrew|, \verb|\JewishDateEnglish|
\item \verb|\GregorianDateHebrew|, \verb|\GregorianDateEnglish|
\item \verb|\publicationinfoHebrew|, \verb|\publicationinfoEnglish|
\end{itemize}

The file \texttt{misc/thesis-fields.tex} contains invocations of several such commands (some of them commented-out with \texttt{\%}), and some additional information about them.

Co-evolution of bacteria and bacteriophages is known as one of the major driving forces in natural ecologies {\color{red} add citations} and an important aspect for potential applications of phages in antimicrobial therapies. {\color{red} add citations} So far, most experimental and computational work focused on tracking co-evolution using well-mixed environments. One of the most prominent results is the rapid overtake of the dynamics by a resistant bacterial genotype outcompeting phage-sensitive ancestors in such a scenario as previously shown {\color{red} add Levin Lenski citation}. Recent experimental and computational results including spatial structured environments exhibit a more complex dynamic namely leading to the coexistence of multiple genotypes and continuous co-evolution of phages and bacteria {\color{red} add Einat and Hitchhiking paper}.

Those works focusing on spatial structure mostly consider either phage plaques, a steady bacteria population invaded by phages or vice versa. To our knowledge, there is only one existing, computational study which focuses on bacterial and phage populations expanding together in space {\color{red} add Hitchhiking paper}. While this study assumes strong chemotaxis, we omit chemotaxis and assume bacteria moving just by diffusion. For this simplified case, a few theoretical studies aim to understand wave propagation based on the theory of expanding waves in a steady state. 
Inspired by previous work {\color{red} add Hitchhiking paper} and the limitations of single agent models regarding size of the possible simulated system {\color{red} add citation}, we focus on partial differential equation models in this work.

\section{General overview}

Over the last decades, bacteria and bacteriophages (from now on phages) became a model system to study host-parasite interactions, however they were lagging behind other systems in understanding their coevolution and ecology~\cite{Koskella2014}.
Despite the early proof that bacteria and phages can coexist in liquid media~\cite{Chao1977}, it was also early on shown that evolution stagnates rapidly in these chemostat cultures~\cite{Lenski1985} and different explanations were proposed why a continuous coevolution is not taking place~\cite{Lenski1984}, recent experiments and models started taking spatial structure of the system's environment into account given the prevalence of spatial structure in natural environments such as biofilms~\cite{Krysiak-Baltyn2016-xi, Gourley2004-rx}. Most work in spatially structured systems focuses on ecological interactions and does not involve possible evolution which might influence the dynamics through evolution. Recent experimental work~\cite{ShaerTamar2022} studied the interaction and evolution of bacteria and phages on agar and showed that evolution can persist in such a scenario.

Combining the recently acknowledged importance of this structure with theory of traveling waves which was recently partly applied to expanding bacteria-phage populations~\cite{Wang2024, Claydon2021}
A more recent attempt to include spatial structure through diffusion in these models was performed by Eriksen et al, 2020 and they use PDEs to model sensitive bacteria, phages and infected bacteria. They compare different models from well-mixed to highly structured environments and conclude that latency time has a huge effect on the survival of bacteria. They also vary the phage inoculum to see how this changes the survival rate of bacteria.
They claim there is no reported case of full bacteria elimination in experiments
Chaudry et al showed experimentally that there exists some "leaky resistance" resulting in transition from resistant to sensitive phenotype or genotype (2018).

There are several proposed mechanisms to guarantee survival of sensitive bacteria and therefore phage survival in a majority resistant population. Chaudry et al nicely summarize them. In well mixed conditions with a specific phage they could show that back mutation can be a leading factor to generate new sensitive bacteria.

Heilmann et al, 2012 showed that at the boundary between two regions (one favorable for phages, one for bacteria), those two can coexist with an interface between them. This interface depends on degradation rate and on infection rate

Simmons et al, 2020 published a very related study modeling phage infection in biofilms composed of sensitive and resistant bacteria promoting coexistence due to small pockets of sensitive bacteria being shielded by the resistant bacteria. Starting with little resistant bacteria leads to an increase of resistant bacteria until they are able to provide the required protection.

Hilborn in 1975 studies the effect of dispersal/diffusion of prey and predator and can show that faster prey and slower predators lead to survival of the prey in space similar to diffusion of phages and bacteria. They do not study resistant prey.

Testa et al (2019) study the influence of spatial structure on phage infection in colonies on agar plates as well as in liquid. They use two Pseudomonas strains (one sensitive to phages, one insensitive) and show that resistance can evolve but not necessarily. Nonetheless, the coexisting insensitive strain can protect the sensitive strain and guarantee its survival without leading to a resistant genotype. They traced this to the mixing and the effective population size at the edge of the colony being relatively small which decreases the chance to obtain a resistant phenotype while also reducing the phage replication and therefore reducing the number of phages. This was verified by finding uninfected bacteria in the colony center. However, when they tried to reproduce this with evolved resistant bacteria, they failed to show that the resistant bacteria provide protection for sensitive bacteria against phage infection and the phage load was not significantly reduced in the center of the colony. It is hypothesized that this is due to the reduced fitness of resistant bacteria. Growth arrest of the bacteria however prevents phages from infecting all sensitive bacteria.

Tzipilevich et al (2017) showed that in liquid, resistant bacteria can obtain receptors from sensitive bacteria by exchange of surface molecules through vesicles.

Brockhurst et al (2006) transferred a bacteria-phage population every 24 hours into a new spatially structured environment to mimic a slowly changing environemnt as an intermediate between constantly shaking liquid and static non-mixed environment. Thy observe coexistence and evolution of bacteria to become resistant but no coevolution, meaning the phages do not evolve in this experiment. However, coexistence is possible in this kind of environment while it is not possible in liquid.

Mitarai et al (2023) showed that a dense bacterial colony provides shielding for bacteria inside the colony leaving the edges unprotected to infection.

Eriksen et al (2024) showed that T4 phages can penetrate deep into a colony and infect bacteria in the center, therefore challenging survival of the community.

Attrill et al (2023) show that nutrient availability for bacteria can change the interactions between phages and E. coli in spatially structured environments. 

Rabinovitch et al (2002) show that phage development in E coli is growth rate dependent and especially burst size changes with growth rate.
However, Golec et al (2014) could show that nonetheless in experiments phages are present in bacteria and can slowly replicate.

Hunter et al (2021) study reaction-diffusion systems with a focus on T7 and the behavior of pulled vs pushed waves. They motivate their model by experiments and then show that their experimental results can be explained by a transition from pulled to pushed wave in the dynamics between bacteria and phages. However, they only consider diffusing phages and ignore the bacteria to be able to diffuse.

\begin{figure}
\includegraphics[scale=0.5]{2024_11_18_phage_inoculation.jpg}
\caption{Here is a caption missing if we include the figure}
\end{figure}

\section{Reaction-Diffusion Models}

\section{Bacteriophages}

\section{Experimental spatial expansion}