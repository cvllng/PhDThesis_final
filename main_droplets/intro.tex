\part{Antibiotic evolution in microdroplets}
\chapter{Introduction}
\label{chap:droplets_intro}

Bacteria can produce various compounds that are toxic to other microorganisms and therefore create antagonistic interactions between those bacteria and microorganisms. These compounds vary greatly in size and can include small molecules, such as antibiotics but also antimicrobial peptides, to large proteins that can be toxic to competitors.

\section{Mimicking natural antibiotic production in a laboratory environment}
A long-term goal in antibiotic research is the challenge of finding conditions in which bacteria evolve to produce novel antibiotics or towards modifying existing ones.~\cite{Charusanti2012-uy} This proves especially difficult as antagonistic interactions such as antibiotic production loose their perceived benefit in commonly used well-mixed conditions. In such conditions, the antibiotic becomes a shared component of the system and therefore is either too diluted to affect the community dynamics or the potential benefit achieved by the producing bacteria can be taken by faster growing resistant "cheating" bacteria~(Fig.~\ref{fig:sketch_benefit}a). Therefore, some kind of spatial structure in the system is necessary. Previous work from our lab~\cite{Ylaine} has shown that it is possible to select for antibiotic producers on agar plates. However, this selection is limited by a critical density above which the selection is hindered by simultaneous selection for cheating bacteria due to proximity to a producing colony.

\section{Microdroplets as an established tool for spatial structure}
Microdroplets 

\section{B. subtilis as a known antibiotic producer}

