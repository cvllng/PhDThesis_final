\part{Selection and evolution of antibiotic producers in microdroplets }
\chapter{Introduction}
\label{chap:droplets_intro}

This chapter introduces the relevant background and previous research for antibiotic evolution, focusing on antibiotic resistance as a global public-health concern and previous efforts to evolve antibiotic producing bacteria as well as providing reasoning, why structured environments are necessary. Furthermore, a section focuses on water-in-oil microdroplets and current technical advancements in this field. The last section describes the scope and research objectives of this project.

\section{Antibiotic resistance is a key threat in public health}

In the last decades, rapidly evolving antibiotic resistance became an increasingly important risk factor in hospital settings and later also in community environments~\cite{David2010-az}. Many efforts have been made to circumvent antibiotic resistance by optimizing treatment regimes~\cite{Kim2014-lq}, repurposing old antibiotics~\cite{Kim2019-qk} or developing entirely new drugs~\cite{Lin2017-sh} but nonetheless antibiotic resistances to clinically relevant antibiotics are commonly found~\cite{Jernigan2020-ro}. We aim to find new ways of overcoming antibiotic resistance by evolving antibiotic-producers, while specifically selecting for the increased killing of resistant bacteria.

\section{Antibiotic production in natural communities}
\label{sec:natural_production}

Natural microbial communities often include producers which gain a growth advantage over their neighboring cells by producing antimicrobial compounds. These antibiotics may give the producer a fitness advantage as they kill or inhibit the growth of its competitors in the community, often allowing the producers to use components of lysed cells as additional nutrients.
As production of an antibiotic is metabolically costly, a producer gains an advantage over neighboring cells if two criteria are fulfilled:
\begin{enumerate}
\item The concentration of the antibiotic is high enough to kill or inhibit the growth of the surrounding competitors.
\item The benefit from killing or inhibiting growth is high enough to out compete resistant, non-producing cells, which we call cheaters.
\end{enumerate}
As cheaters are not affected by the produced antibiotic but also do not produce the specific antimicrobial compounds, they may have a growth advantage over the producer and can therefore overtake the free space generated by the producer ("cheating") limiting its fitness advantage.
A long-term goal in antibiotic research is the challenge of finding conditions in which bacteria evolve to produce novel antibiotics or towards modifying existing ones, effectively emulating such natural communities.~\cite{Charusanti2012-uy} 

\section{Selection is not possible in well-mixed environments}

The effect of an antimicrobial compound in spatially structured environments is limited by diffusion to the immediate surrounding of the producers. Therefore one only sees a growth inhibition zone in close proximity to a colony of producers. One could think that using a well mixed environment could increase the effect of the antibiotic.
However, due to the constant mixing in a well-mixed environment, the produced antibiotic is usually dispersed through the entire culture and does not accumulate to a high enough concentration required to actually kill competitors as sketched (not to scale) in figure~\ref{fig:intro_droplets_idea}a. Moreover, even if the antibiotic accumulates to a high enough concentration (due to high production rates or a large number of producing cells) the chance of an existing cheater growing equally fast or even faster than the producer is close to 1.
Therefore, according to the criteria defined in section~\ref{sec:natural_production}, producers do not have a growth advantage over other cells in well mixed environments and are usually not selected for or even out-competed by cheaters due to the significantly lower growth rate as a result of the antibiotic production.

\section{Spatial structures enable selection for antibiotic producers}
\label{sec:spatial_structure}

Agar plates as spatially structured environments were successfully used in a previous study to show that producers have a growth advantage compared to cheaters when grown in competition with sensitive cells (the colonies of producers are bigger than the ones of cheaters). However, as soon as a cheater is in close proximity to a producer (within the inhibition zone of the latter one), it will profit from the killing of the sensitive cells. Therefore one can only work with very low densities of producers on agar plates to minimize the probability of a cheater to be in the inhibition zone created by a producer~\cite{Gerardin2016-ac}. The small population size available in this setting limits the potential for de novo mutations which would lead to evolution of the antibiotic producer.

\begin{figure}
\includegraphics[width=\linewidth]{graphics/2025_09_30_droplets_fig1.png}

\caption{\textbf{Microdroplets provide a platform for selection of antibiotic producing cells.} In this sketch, the interactions between sensitive (blue), producing (dark red) and cheating (light red) cells in a well-mixed (a) and in a water-in-oil droplet (b) environment are shown. \textbf{(a)} Upon production of the antibiotic (red crosses), it is immediately dispersed through the entire culture due to constant mixing. Two possible outcomes are possible, as sketched here. In the first outcome, the antibiotic is too diluted to inhibit growth (as shown in the upper part) and all cells grow at a similar rate. In the second outcome, the antibiotic is highly concentrated and kills the sensitive cells in the entire culture as shown in the lower part. Due to the killing of all sensitive cells, the cheater cells can equally profit from the absence of the sensitive cells and both, producers and cheaters, grow. \textbf{(b)} In droplets producers and cheaters can be separated in different droplets as depicted here. The producer will target sensitive cells trapped in the same droplet as the antibiotic molecules cannot disperse through the entire culture (into other droplets). So the producer is the only cell which profits from the effect of the antibiotic and grows faster than cheaters and sensitive cells in competition in other droplets as depicted in the last step.}
\label{fig:intro_droplets_idea}
\end{figure}

\section{Water-in-oil microdroplets as a technology for \\ high-throughput spatially structured microenvironments}

In recent years, microdroplets have become a commonly used technology in biomedical applications~\cite{Zagnoni2011-ko}. They enable us to encapsulate small volumes $ \left( \approx 50 \mathrm{pl} \right)$ of growth medium and competing cells in separate microenvironments as sketched in figure \ref{fig:intro_droplets_idea}b. Due to the high generating frequency in microfluidic chips, currently accessible $\left( \approx 10^4 \frac{\mathrm{droplets}}{\mathrm{s}} \right) $, we can now perform high-throughput experiments.
Therefore, with microdroplets, we can replace the problem mentioned in section~\ref{sec:spatial_structure} of a limiting density of producers on agar plates with the problem of a limiting concentration of producers in the aqueous phase which can more easily be dealt with by producing more droplets, which is a matter of time rather than a matter of space as it was on agar plates~\cite{Gerardin2016-ac}. Having longer production times for droplets would increase the number of produced droplets by up to $10^4$ droplets per second.
Assuming the chance for a mutation which modifies antibiotic production to be $10^{-8}$~\cite{Drake1998-es, Kunkel2004-br, Wielgoss2011-jd} and our ability to easily reach $10^7$ droplets in a reasonable time, we can considerably increase the chance of obtaining such mutations compared to agar plates used previously, where the maximal number of distinct colonies reached $10^3$~\cite{Gerardin2016-ac}.

\section{Scope and objectives of the project}

The project has three main objectives:
(1) We will establish a high-throughput microdroplet system to select for antibiotic producing cells.
(2) We will select for the antibiotic producing cells in competition with sensitive and cheater cells in microdroplets.
(3) We will find evolutionary pathways leading to de novo mutations within the antibiotic producing cells which help overcoming evolved resistance.
By evolving the antibiotic to target resistant cells, we would be able to isolate the antibiotic from its producer in a next step as done before~\cite{Singh2018-qp} and one could conduct clinical studies to use this antibiotic in patients treatment. The project would serve as a proof-of-principle how one could try to target the increasing threat of antibiotic resistances.
The resulting mutations could offer a general way to improve antibiotics effectiveness which is not known yet and therefore could lead to new opportunities of generating new, enhanced antibiotics which can be used to treat infections with resistant bacterial strains.
We expect this project to have a significant impact on how we currently think about evolvability of antibiotic producing cells and how these obtain de novo mutations which give a fitness advantage. Furthermore, we expect to gain insight into the dynamics of an arms race between antibiotic producing and resistant cells and therefore understand the key principles of how resistance evolves in the presence of antibiotic producing cells.

