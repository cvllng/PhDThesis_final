\part{Antibiotic evolution in microdroplets}
\chapter{Introduction}
\label{chap:droplets_intro}

\section{Mimicking natural antibiotic production in a laboratory environment}
A long-term goal in antibiotic research is the challenge of finding conditions in which bacteria evolve to produce novel antibiotics or towards modifying existing ones.~\cite{Charusanti2012-uy} This proves especially difficult as antagonistic interactions such as antibiotic production loose their perceived benefit in commonly used well-mixed conditions. In such conditions, the antibiotic becomes a shared component of the system and therefore is either too diluted to affect the community dynamics or the potential benefit achieved by the producing bacteria can be taken by faster growing resistant "cheating" bacteria~(Fig.~\ref{fig:sketch_benefit}a). Therefore, some kind of spatial structure in the system is necessary. Previous work from our lab~\cite{Ylaine} has shown 

\section{Microdroplets as an established tool for spatial structure}

\section{B. subtilis as a known antibiotic producer}

